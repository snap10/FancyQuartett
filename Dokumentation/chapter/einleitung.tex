\chapter{Einleitung}
\label{cha:einleitung}

Seit der Ankündigung des weltweit ersten Smartphones durch \textit{Apple Inc.} im Jahre 2007 hat sich sehr viel auf dem Markt für mobile Endgeräte getan. Heute gibt es eine Vielzahl an verschiedenen Smartphones von unterschiedlichen Herstellern, die jeweils mit unterschiedlichen Betriebssystemen ausgestattet sind, wie z.B. \textit{Android} von \textit{Google Inc.}, \textit{iOS} von \textit{Apple Inc.} oder \textit{WindowsPhone} von \textit{Microsoft}. Smartphones sind heute nicht mehr aus unserem Alltag wegzudenken, da sie den Benutzer aufgrund ihrer Mobilität und dank verschiedener Applikationen (kurz \glqq Apps\grqq) unterstützen aber auch unterhalten. Im diesjährigen Anwendungsfachs \textit{Mobile Application Lab} entstand deshalb im Rahmen eines kleinen Softwareprojektes eine Quartett-App, die sich durch eine beliebige Anzahl an auswählbaren Kartendecks von den bereits existierenden Apps im App-Store abheben soll. Hierbei werden Kartendecks per \textit{Representational State Transfer (REST)} von einem Server der Universität Ulm heruntergeladen, die dann auf dem Gerät persistent gespeichert werden und somit auch offline verfügbar sind. Neben den vorgegebenen Anforderungen durfte jedes Team eigene Ideen einbringen, was zu unterschiedlichen Anwendungen führte.

\section{Motivation}
\label{sec:motivation}

Apps spielen heute, aber auch in Zukunft eine große Rolle im mobilen Software-Bereich. Durch die Programmierung einer Quartett-App in Teams soll mehr Praxiserfahrung in das doch eher theorielastige Studium mit einfließen. Die Quartett-App deckt möglichst viele Bereiche der App-Programmierung ab, wodurch die Teams später in der Lage sind, ihre eigenen Ideen und Anwendungen planen und umsetzen zu können.

\section{Kontext}
\label{sec:kontext}

Die Quartett-App wird im Auftrag des \textit{Institut für Datenbanken und Informationssysteme} der \textit{Universität Ulm} entwickelt. Dabei sollen die Teams grundlegende Werkzeuge zur App-Programmierung kennenlernen und somit Schritt für Schritt die benötigten Schritte der App-Entwicklung meistern. Durch sorgfältige Planung, wie z.B. mit Hilfe von Mockups oder Diagrammen, soll die spätere Implementierung vereinfacht werden. Neben der Planung und Implementierung der Anwendung sollen die Teams sich mit den Besonderheiten der Zielplattform und der dazugehörigen Smartphones (wie z.B. Datenhaltung, Sensorik oder gestenbasierter Eingabe) auseinander setzen.


