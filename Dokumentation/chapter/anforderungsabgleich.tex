\chapter{Anforderungsabgleich}
\label{cha:anforderungsabgleich}

In der folgenden Tabelle soll verdeutlicht werden, wie gut die an die App gestellten Anforderungen umgesetzt wurden. Die verwendete Skala geht von \glqq sehr gut\grqq\ (1) bis \glqq ungenügend\grqq\ (6). Außerdem wird im Folgenden noch einmal auf jede Anforderung konkret eingegangen.

\section{Funktionale Anforderungen}
\label{sec:abgleich_funtionaleanforderungen}

\begin{table}[ht]
\centering
\begin{tabular}{l|c c c c c c}
funktionale Anforderungen & 1 & 2 & 3 & 4 & 5 & 6 \\ \hline\hline
Spieltyp & $\checkmark$ &  &  &  &  &   \\
Spielmodi & $\checkmark$ &  &  &  &  & \\
Schwierigkeitsgrade & $\checkmark$ &  &  &  &  & \\
Einstellungsmöglichkeiten & $\checkmark$ &  &  &  &  & \\
Gespeichertes Spiel & $\checkmark$ &  &  &  &  & \\
Statistiken & $\checkmark$ &  &  &  &  & \\
Galerie & $\checkmark$ &  &  &  &  & \\
Kartendecks & $\checkmark$ &  &  &  &  & \\
Serverkommunikation &  & $\checkmark$ &  &  &  &
\end{tabular}
\label{tab:abgleich_funktionaleanforderungen}
\end{table}

\subsection{Spieltyp}
Der Benutzer kann vor beginn eines Spiels wählen, ob er ein reines Single-Player-Spiel oder ein Hotseat-Spiel gegen einen Freund spielen möchte. Außerdem wird ein \emph{Zuschauermodus} zur Auswahl angeboten, in dem der Spieler einem Duell zwischen 2 Computergegnern zuschauen kann.

\vspace{5mm}
\emph{Bewertung: sehr gut}
\vspace{5mm}

\subsection{Spielmodi}
Dem Benutzer werden die Spielmodi \emph{To-The-End} und \emph{Time} zur Auswahl angeboten. Der selbst entwickelte Spielmodus \emph{Points} soll die Spieler dazu verleiten auch \emph{schlechte} Karten zu spielen. Durch die unterschiedlichen Spielmodi soll die Langzeitmotivation des Spielers gefordert werden.

\vspace{5mm}
\emph{Bewertung: sehr gut}
\vspace{5mm}

\subsection{Schwierigkeitsgrade}
Die verschiedenen Schwierigkeitsgrade entsprechen alle ihrer Beschreibung. Gegen einen leichten Computergegner gelingt es fast immer zu gewinnen, wohingegen der schwere Computergegner den Spieler wirklich fordert und somit Herausforderung darstellt.

\vspace{5mm}
\emph{Bewertung: sehr gut}
\vspace{5mm}

\subsection{Einstellungsmöglichkeiten}
Vor jedem Spiel hat der Benutzer die Möglichkeit die \emph{maximale Rundenzahl} und/oder ein \emph{Zeitlimit für einen Spielzug} zu aktivieren und einzustellen. Dadurch kann der Benutzer selbst entscheiden ob das Spiel lang oder kurz dauert.

\vspace{5mm}
\emph{Bewertung: sehr gut}
\vspace{5mm}

\subsection{Gespeichertes Spiel}
Verlässt der Benutzer ein gerade laufendes Spiel, so bekommt er eine Meldung, dass im Hintergrund das Spiel gespeichert wird und er jederzeit das Spiel fortsetzen kann. Das Spiel wird dann im Hintergrund persistent gespeichert. Sobald das Spiel fertig gespielt wurde, wird der Spielstand gelöscht.

\vspace{5mm}
\emph{Bewertung: sehr gut}
\vspace{5mm}

\subsection{Statistiken}

Der Spieler kann in der \emph{MainActivity} die Statistiken einsehen. Hierbei werden die Anzahl der bisher gespielten Spiele und die Anzahl der bisherigen Duelle aufgelistet, welche jeweils die Anzahl der Siege bzw. Niederlagen beinhalten.

\vspace{5mm}
\emph{Bewertung: sehr gut}
\vspace{5mm}

\subsection{Galerie}

In der Galerie kann der Spieler alle verfügbaren Kartendecks sehen - sowohl die lokal gespeicherten als auch die auf dem Server bereit liegenden. Die online verfügbaren Decks sind mit einem \emph{Download-Icon} gekennzeichnet.

Bei Auswahl eines lokalen Decks werden dessen Karten angezeigt. Bei Auswahl einer Karte werden deren Bilder und Attributwerte angezeigt.

Bei Auswahl eines online verfügbaren Decks wird der Download gestartet. Nach dessen Beendigung verschwindet das \emph{Download-Icon}.

Über das Menü eines jeden Decks kann ein neues Spiel mit diesem Deck gestartet oder das Deck vom Gerät gelöscht werden. Nach Auswahl von \emph{Neues Einzelspieler} oder \emph{Neues Mehrspieler} wird die Ansicht für die Spieleinstellungen angezeigt. Nach Auswahl von \emph{Deck löschen} werden die Daten im Dateisystem gelöscht und das Deck steht anschließend wieder zum Download bereit, falls es auf dem Server noch verfügbar ist.

\vspace{5mm}
\emph{Bewertung: sehr gut}
\vspace{5mm}

\subsection{Serverkommunikation}

Beim Download eines Decks wird ein Fortschrittsbalken mit einer Prozentzahl angezeigt, um dem Nutzer zu zeigen, wann der Download beendet ist. Die App lädt Karten und Bilder sequentiell, d.h. es wird ein HTTP Request nach dem anderen gesendet. Dies ist ein praktikables Vorgehen, allerdings besteht Optimierungspotenzial. Die Karten sowie deren Bilder sind voneinander unabhängig und können daher parallel geladen werden. Ein paralleles Aufbauen mehrerer TCP-Verbindungen würde die Netzwerklast erhöhen und die verfügbare Bandbreite stärker ausnutzen.

Die Geschwindigkeit des Downloads könnte somit verbessert werden. Jedoch dauert ein Deck-Download bei den derzeit verfügbaren Decks selbst im mobilen Datennetz nicht mehr als 2 Minuten. Die Fortschrittsanzeige beugt möglicher Frustration vor. Daher kann die Serverkommunikation als gut bewertet werden.

\vspace{5mm}
\emph{Bewertung: gut}
\vspace{5mm}

\section{Nicht-funktionale Anforderungen}
\label{sec:abgleich_nichtfunktionaleanforderungen}

\begin{table}[ht]
\centering
\begin{tabular}{l|c c c c c c}
funktionale Anforderungen & 1 & 2 & 3 & 4 & 5 & 6 \\ \hline\hline
Material Design & $\checkmark$ &  &  &  &  &   \\
Gestensteuerung &  & $\checkmark$  &  &  &  & \\
Animationen & $\checkmark$ &  &  &  &  & \\
Offline-Szenario & & $\checkmark$  &  &  &  &
\end{tabular}
\label{tab:abgleich_nichtfunktionaleanforderungen}
\end{table}

\subsection{Material Design}

Das Design der App ist streng an den Standard Design Bibliotheken von Android gehalten und erfüllt damit die Anforderungen des Material Designs. Alle verwendeten Icons sind Teil des Material Design Paketes. Die verwendeten Konzepte wie Viewpager mit integrierten 
Tabview Indikatoren sowie die von Android bekannte Backstack Navigation ermöglichen dem Nutzern ein gewohntes Navigieren mit bekanntem Look and Feel seiner Platform. Kontextmenüs sind gewohnt durch das typische Menüicon als auch durch Long-Press zu öffnen. Die Gallerielisten ermöglichen das ebenfalls bekannte Umschalten zwischen Listen und Kachelansicht mit den gewohnten Icons in der Actionbar.
% TODO Ferdi@all: fehlt noch was???
\vspace{5mm}
\emph{Bewertung: sehr gut}
\vspace{5mm}

\subsection{Gestensteuerung}
Die Anwendung unterstützt die auf Touchgeräten wohl wichtigste Geste, dass Wischen direkt in der MainActivity, wo sie das leichte Navigieren zwischen den drei Hauptfunktionen Spielen, Gallerie und Statistik ermöglicht. Zusätzlich lässt sich in den Gallerielisten die Wischgeste zum gewohnten Scrollen einsetzen. Ganz intuitiv kann die Wischgeste in der Kartenansicht verwendet werden um seitlich durch die Karten zu blättern. 
An dieser Stelle hätte man eventuell noch eine Kippgeste für die Blätterfunktion implementieren können.
Die Schüttelgeste ermöglicht in den Spieleinstellungen ein schnelles zufälliges Einstellen der Spielattribute. Dadurch muss sich der Spieler nicht lang mit Einstellen aufhalten, sondern kann direkt mit zufälligen Einstellungen starten.
\vspace{5mm}
\emph{Bewertung: gut}
\vspace{5mm}

\subsection{Animationen}
Die Anwendung nutzt lediglich dezente Animationen um den Bedingungen des Material Designs gerecht zu werden. Zum einen werden Fehlbedienungen und Inkorrekte Anwendungszustände durch dynamische Toasts realisiert die kurzzeitig eingeblendet werden und somit nicht störend kleine aber wichtige Informationen bereitstellen. 
Alle Dialoge wurden durch eine eigens erstellte Animation erweitert die sie von unten in den Bildschirm fahren lassen. Der Nutzer wird dadurch nicht von einem plötzlich erscheinenden Dialog überrascht sondern durch die kurzzeitige Animation unterschwellig auf den Dialog vorbereitet. Insgesamt werden durch die dezenten Animationen ein geschmeidiges Look and Feel erzeugt.

\vspace{5mm}
\emph{Bewertung: sehr gut}
\vspace{5mm}

\subsection{Offline-Szenario}
Die grundsätzlichen Anforderungen für Offline Verhalten werden auf funktionaler Seite vollständig erfüllt. Der Nutzer kann alle heruntergeladenen Decks auch ohne Internetverbindung nutzen und ohne Einschränkungen damit spielen. Ein möglicher Nachteil könnte sein, dass die App bei der Installation ohne spielbares Offlinedeck ausgeliefert wird. Es ist zwingend erforderlich selber ein Deck herunterzuladen. Da aber zum Installieren aus dem Store eine Internetverbindung vorhanden sein muss ist der Nutzer höchstwahrscheinlich auch direkt in der Lage ein Deck herunterzuladen. 

\vspace{5mm}
\emph{Bewertung: gut}
\vspace{5mm}

